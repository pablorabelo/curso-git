\documentclass[12pt,openright]{report}
\topmargin-.9cm
\oddsidemargin.4cm

\usepackage[brazil]{babel}
\usepackage[utf8]{inputenc} %pacote de acentuação
\usepackage{amsmath}
\usepackage{amsfonts}
\usepackage{amssymb}
\usepackage{amsthm, amstext, amscd}
%\left( \usepackage[dvips]{graphicx}
\usepackage[all]{xy}
\usepackage{graphicx}
\usepackage{lscape}
\usepackage{geometry}
\usepackage{indentfirst}
\usepackage{longtable}
%\usepackage{caption}
\usepackage[justification=centering]{caption} % centralizar título das figuras e tabelas
\usepackage{float}
\usepackage{natbib}
\usepackage{url}
\usepackage[table]{xcolor}
\pagestyle{myheadings} %número de páginas no topo à direita
%\pagestyle{plain} %número de páginas no rodapé e no centro
%\ProvidesFile{actuarialsymbol.dtx}
%\usepackage{microtype}
% \usepackage{rsfso}            % \mathcal variants
%\usepackage{actuarialsymbol}[2017/04/10]
%\usepackage{abnteX}

%\usepackage[center,small]{caption}

\geometry{paperwidth=210mm,paperheight=297mm,
textwidth=180mm,textheight=240mm,
top=20mm,bottom=20mm,left=18mm,right=18mm}

\evensidemargin 0.2cm


%\setlength{\textwidth}{16cm} \setlength{\textheight}{24cm}
%\setlength{\hoffset}{-1mm} \setlength{\voffset}{-20pt}


\renewcommand {\baselinestretch}{1.5}
\newtheorem{theorem}{Teorema}[section]
\newtheorem{corollary}[theorem]{Corolário}
\newtheorem{definition}[theorem]{Definição}
\newtheorem{example}[theorem]{Exemplo}
\newtheorem{proposition}[theorem]{Proposição}
\newtheorem{remark}[theorem]{Observação}
\newtheorem{summary}[theorem]{Summary}
\newtheorem{theorems}{Teorema}[subsection]
\newtheorem{acknowledgements}[theorems]{Acknowledgement}
\newtheorem{definitions}[theorems]{Definição}
\newtheorem{propositions}[theorems]{Proposição}
\newtheorem{remarks}[theorems]{Observação}
\newtheorem{teorema}{Teorema}[section]
\newtheorem{Lem}{Lema}[section]
%\newenvironment{proof}[1][Prova]{\textbf{#1.} }{\ \rule{0.5em}{0.5em}}



\begin{document}
jfibjabfvjiafbvjanfvjfnvjnfvjnvnk fjnjfvnjfv fjvnjfnvjifnvfj jnjfnvjfnjnfvj vnfjnvjfnjfnv njfnvjfnvjfnvjkf fnjfnjbfjkbakjfv f vjgkjanfkjvnfjhv afkjbngkjaf akjdfbkjabfkjbf. 

novo parágrafo 

o curso %é bem interessante %

 é legal "por exemplo" ``por exemplo"




\newpage

\begin{center}
	{\LARGE 	\textbf{Algumas formatações de texto}}
\end{center}

Notas de rodapé e notas de lado

O comando footnote \footnote{insere o texto no rodapé}.

a UNIFAL \footnote{universidade federal de alfenas}

\begin{enumerate}
	\item Você pode misturar os
	ambientes de lista ao seu gosto:
	\begin{itemize}
		\item Mas eles podem ter uma
		aparência melhor.
		\item[-] Com um hífen.
	\end{itemize}
	\item Entretanto lembre-se:
	\begin{description}
		\item[Coisas inúteis] não se tornarão
		úteis porque estão em uma lista.
		\item[Coisas úteis], entretanto, podem
		ser bem apresentadas em uma lista.
	\end{description}
\end{enumerate}
\newpage
\begin{center}
	{\LARGE \textbf{Alguns exemplos matemáticos}}
\end{center}

% Exemplo 1
\ldots quando Einstein introduziu sua fórmula
\begin{equation}
e = m \cdot c^2 \; ,
\end{equation}
que é ao mesmo tempo a mais conhecida
e a menos compreendida fórmula da física.








$ 1 $ 1

$ A_b $











% Exemplo 2
\ldots do que segue a lei de Kirchoff:
\begin{equation}
\sum_{k=1}^{n} I_k = 0 \; .
\end{equation}
A lei da voltagem de Kirchhoff pode ser derivada \ldots

\begin{center}
	{\LARGE \textbf{Fração, raiz, números elevados e sub escritos}}
\end{center}

$\displaystyle \frac{1}{2} \ \, \ \ \sqrt{4}, \ \ 2^2, \ \  Y_i$

$$ \int_{0}^{\infty} x^2 dx $$

\begin{center}
	{\LARGE \textbf{Alinhando fórmulas matemáticas}}
\end{center}

$
y = \left\{ \begin{array}{ll}
	a & \textrm{se $d>c$}\\
	b+x & \textrm{de manhã}\\
	l & \textrm{o resto do dia}
\end{array} \right.
$


\begin{eqnarray}
f(x) & = & \cos x \\
f'(x) & = & -\sin x \\
\int_{0}^{x} f(y)dy &
= & \sin x
\end{eqnarray}

$
\mathop{\mathrm{corr}}(X,Y)=
\frac{\displaystyle
	\sum_{i=1}^n(x_i-\overline x)
	(y_i-\overline y)}
{\displaystyle\biggl[
	\sum_{i=1}^n(x_i-\overline x)^2
	\sum_{i=1}^n(y_i-\overline y)^2
	\biggr]^{1/2}}
$

\newpage
{\LARGE \textbf{Inserindo Tabelas}}








https://www.tablesgenerator.com/
\newpage
\textbf{{\LARGE Inserindo Figuras}}
\begin{figure}[H]
	\centering
	\includegraphics[width=0.1\linewidth]{LIGA}
	\caption{}
	\label{fig:liga}
\end{figure}

\newpage
{\LARGE \textbf{Desenhando}}

\begin{picture}(50,25)
\put(20,5){\circle{20}}
\put(20,5){\vector(0,1){10}}
\put(50,5){\circle*{5}}
\end{picture}


\newpage
{\LARGE \textbf{Alguns comandos úteis para Trabalhos acadêmicos}}
%\newpage
%\tableofcontents
%\thispagestyle{empty}
%\newpage
%\listoffigures
%\thispagestyle{empty}
%\newpage
%\listoftables
%\thispagestyle{empty}
\newpage
Alguns acadêmicos da \textbf{Universidade Federal de Alfenas} – Campus Avançado de Varginha se juntaram para desenvolver um projeto \underline{{\LARGE inédito}} no país, a \textit{Liga de Ciências Atuariais (LCA)}. 

\begin{figure}[H]
	\centering
	\includegraphics[width=0.3\linewidth]{LIGA}
	%\caption{}
	\label{fig:liga}
\end{figure}

No âmbito atuarial existem diversos conceitos e fórmulas, como por exemplo a famosa notação para o valor presente atuarial do seguro de vida temporário é dado por \footnote{$_{k}p_{x}q_{x+k}$ são probabilidade de sobrevivência e morte respectivamente, retiradas da tábua de vida}:

\begin{equation} A^{1}_{x:n \rceil} = E[Z]= B \sum_{k=0}^{n-1} v^{k+1} \ \ _{k}p_{x}q_{x+k}
\label{SegVidaTemp} 
\end{equation}

Alguns exemplos:

\begin{table}[H]
	\centering
	\caption{Valores parciais do somatório para o cálculo do prêmio puro único de $A^{1}_{60:5 \rceil}$}
	\label{my-label}
	\begin{tabular}{@{}ccc@{}}
		\hline
		$k$                     & $v^{k+1}$       & $_{k}p_{x} \ \ q_{x+k}$    \\ \hline
		0                     & 0,9672115 & 0,00886 \\
		1                     &0,9354981 & 0,00955 \\
		\multicolumn{1}{r}{2} & 0,9048246 & 0,01030 \\
		\multicolumn{1}{r}{3} & 0,8751568 & 0,01110 \\
		4                     & 0,8464617 & 0,01196 \\ \hline
		\label{pq}
	\end{tabular}
\end{table}

\end{document}